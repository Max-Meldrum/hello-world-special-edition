%
% File paper.tex


\documentclass[11pt]{article}
\usepackage{acl2015}
\usepackage{times}
\usepackage{url}
\usepackage{latexsym}
\usepackage{natbib}


\usepackage[noend]{algpseudocode}
\usepackage{amsmath}
\usepackage{algorithm}
\usepackage{pifont}

%\setlength\titlebox{5cm}

% You can expand the titlebox if you need extra space
% to show all the authors. Please do not make the titlebox
% smaller than 5cm (the original size); we will check this
% in the camera-ready version and ask you to change it back.


\title{Hello World: Special Edition}

\author{Max Meldrum \\
{\tt max@meldrum.se} \\\And
Second Author \\
{\tt email@domain} \\}


\makeatletter
\def\BState{\State\hskip-\ALG@thistlm}
\makeatother

\date{}

\begin{document}
\maketitle
\begin{abstract}
Since being introduced in the book "The C Programming Language", Hello World has become the typical way of how people are introduced to the world of programming. We observe that the classical approach is conservative, all characters are displayed solely through one programming language. In this paper, we propose an improved version which enables the execution to transpire through multiple languages.
\end{abstract}

\section{Introduction}
Instructing the computer to display text on the screen is the most common first program that any developer will encounter. The concept of Hello World became popular through\cite{Kernighan:1988:CPL:576122}. 

\section{Algorithm}
In our implementation, we have opted to go for "Hello World!" instead of "Hello, World!". In total, including the white space, we have 12 characters. Any number of languages is supported, as long it is able to compile on the target computer. It is not guaranteed that the 12 selected languages are unique, the algorithm will pick randomly from a basket.
\begin{algorithm}
\caption{Initialization Phase}
\label{Special Edition}
\begin{algorithmic}[1]
\Procedure{GenerateLanguages}{}
\State $N\gets 12$
\State $L\gets Languages$
\For{$N$ iterations}
	\State $S\gets Random(L)$
\EndFor
\State \textbf{return} S
\EndProcedure
\end{algorithmic}
\end{algorithm}

\subsection{Executor}
The Executor program has two tasks. PartialPrint and Notify. The former checks if the given character position is valid, if it is, it then prints the character that corresponds to the position. The latter reviews the current position and decides whether to pass on the task to the next Executor in line or exit the system.

\begin{algorithm}
\caption{Executor System}
\label{Special Edition}
\begin{algorithmic}[2]
\Procedure{PartialPrint(pos, nlangs)}{}
\If{pos == 0} print "H"
\ElsIf{pos == 1} print "e"
\ElsIf{pos == 2} print "l" 
\ElsIf{pos == 3} print "l"
\ElsIf{pos == 4} print "o"
\ElsIf{pos == 5} print " "
\ElsIf{pos == 6} print "W"
\ElsIf{pos == 7} print "o"
\ElsIf{pos == 8} print "r"
\ElsIf{pos == 9} print "l" 
\ElsIf{pos == 10} print "d"
\ElsIf{pos == 11} print "!"
\ElsIf{pos == 12} print "\texttt{\char`\\n}"
\EndIf
\EndProcedure

\Procedure{Notify(pos, nlangs)}{}
\If{pos \texttt{\char`\>=} 0 and pos \texttt{\char`\<} 12} \State $target\gets nlangs.head$
\Else {    return 0}
\EndIf
\State $executors\gets nlangs.dropHead$
\State Controller(target,  pos+1, executors)
\EndProcedure
\end{algorithmic}
\end{algorithm}

\subsection{Controller}
The controller acts as a bridge between Executor's. After an Executor has printed a character, it notifies the controller of the next target language, position and the list that contains the Executor order.


\section{Evaluation}

\section{Future Work}

\section{Conclusion}

\section*{Acknowledgments}


\bibliography{references}
\bibliographystyle{unsrt}

\end{document}
